\documentclass[fleqn,twoside]{scrartcl}
\usepackage[utf8]{inputenc}

\usepackage[aux]{rerunfilecheck}
\usepackage[main=ngerman]{babel}

\usepackage{amsmath}
\usepackage{amssymb}
\usepackage{mathtools}

\usepackage{fontspec}

\usepackage[
  math-style=ISO,
  bold-style=ISO,
  sans-style=italic,
  nabla=upright,
  partial=upright,
]{unicode-math}
\setmathfont{Latin Modern Math}

\usepackage[autostyle]{csquotes}

\usepackage[
  locale=DE,
  separate-uncertainty=true,
  per-mode=symbol-or-fraction,
]{siunitx}

\usepackage{xfrac}

\usepackage[
  labelfont=bf,
  font=small,
  width=0.9\textwidth,
]{caption}
\usepackage{subcaption}

\usepackage{graphicx}

\usepackage{float}
\usepackage{scrhack}
\floatplacement{figure}{htbp}
\floatplacement{table}{htbp}

\usepackage{booktabs}
\usepackage{multirow}

\usepackage{xparse}
\usepackage{xstring}
\usepackage{xcolor}
\xdefinecolor{tugreen}{RGB}{128, 186, 38}
\colorlet{tudark}{tugreen!60!black}

\usepackage{microtype}
\usepackage[top=2cm,left=1.5cm,right=1.5cm]{geometry}

\usepackage{enumitem}
\setlist[enumerate,1]{label=\bfseries\alph*), parsep=0pt}

\usepackage[
  colorlinks,
  urlcolor=tudark,
  unicode,
  pdfcreator={},
  pdfproducer={},
]{hyperref}
\usepackage{bookmark}

\usepackage[shortcuts]{extdash}
\usepackage{tikz}
\usetikzlibrary{calc}
\usetikzlibrary{arrows}
\usetikzlibrary{arrows.meta}
\usetikzlibrary{patterns}
\usetikzlibrary{quotes,angles}
\usetikzlibrary{circuits.ee.IEC}
\usetikzlibrary{positioning}
\usetikzlibrary{decorations.pathmorphing}
\usetikzlibrary{decorations.pathreplacing}
\usepackage{framed}

\setkomafont{pageheadfoot}{\normalfont}
\markboth{}{Name: \underline{\hspace{8.1cm}} Matrikelnummer: \underline{\hspace{5.6cm}}}
\pagestyle{myheadings}

%%% Settings
\newcommand{\thesemester}{Wintersemester 2019/2020}
\newcommand{\theprofessor}{Prof.~Dr.~Maximilian~Mustermann}
\newcommand{\thedate}{31. Februar 1234}

\newcommand{\pkt}[1]{\textbf{#1\,P.}}

\newcounter{exercise}
\newenvironment{exercise}
[2]
{\addtocounter{exercise}{1}{\Large{\bfseries{Aufgabe \arabic{exercise}:~#1}\hfill(#2 Punkte)}}\parskip5ex

}
{\medskip}
\newif\ifdisplaysolutions
\IfStrEq{\jobname}{\detokenize{solution}}{% True branch
  \displaysolutionstrue
}{}

\newenvironment{solution}
{
	\ifdisplaysolutions ~\\ {\Large{\bfseries{Lösung zu Aufgabe \arabic{exercise}:}\\\parskip5ex}}
  \else
  	\setbox0\vbox\bgroup
  \fi
}
{\ifdisplaysolutions
  \medskip
  \else
  	\egroup
  \fi
}

\setlength{\parindent}{0mm}

\usepackage{mleftright}
\DeclarePairedDelimiter{\bra}{\langle}{\rvert}
\DeclarePairedDelimiter{\ket}{\lvert}{\rangle}
\DeclarePairedDelimiterX{\braket}[2]{\langle}{\rangle}{#1 \delimsize| #2}

\usepackage{expl3}
\ExplSyntaxOn
\NewDocumentCommand \dif {m} {\mathinner{\mathrm{d} #1}}
\NewDocumentCommand \del {m} {\mathinner{\mathrm{δ} #1}}
\NewDocumentCommand \Del {m} {\mathinner{\mathrm{Δ} #1}}
\NewDocumentCommand \matrize {m} {\underline{\underline{#1}}}
\NewDocumentCommand \abs {m} {\left|#1\right|}
\NewDocumentCommand \ableitung {mm} {\frac{\partial #1}{\partial #2}}
\NewDocumentCommand \abl {mm} {\frac{\dif{#1}}{\dif{#2}}}
\NewDocumentCommand \I {} {\mathrm{i}}
\NewDocumentCommand \E {} {\mathrm{e}}
\NewDocumentCommand \mpi {} {\symup{π}}
\NewDocumentCommand \grau {m} {\textcolor{gray}{#1}}
\NewDocumentCommand \rot {m} {\textcolor{red}{#1}}
\NewDocumentCommand \tug {m} {\textcolor{tugreen}{#1}}
\NewDocumentCommand \zB {} {z.\,B.~}
\NewDocumentCommand \DaH {} {d.\,h.~}
\NewDocumentCommand \IN {} {^{-1}}
\NewDocumentCommand \punkte {m} {\textcolor{red}{\textit{(#1)}}}
\let\ltext=\l
\RenewDocumentCommand \l {} {\TextOrMath{ \ltext }{ \mleft }}
\let\raccent=\r
\RenewDocumentCommand \r {} {\TextOrMath{ \raccent }{ \mright }}
\ExplSyntaxOff
\NewDocumentCommand \zZ {} {\textsf{Z\kern-.49em\raise-0.65ex\hbox{Z}}}

% See: https://alexwlchan.net/2017/10/latex-underlines/
\usepackage{contour}
\usepackage{ulem}
\renewcommand{\ULdepth}{1.8pt}
\contourlength{0.8pt}
\newcommand{\myuline}[1]{%
  \uline{\phantom{#1}}%
  \llap{\contour{white}{#1}}%
}

\newcommand{\leerseite}{\newpage\hspace{1cm}\newpage}


\begin{document}
\DeclareGraphicsRule{*}{mps}{*}{}
\begin{titlepage}
	\begin{center}
		\vspace*{0cm}
    %%% Settings
		{\Huge\bfseries Abschlussklausur zur Vorlesung}\smallskip\\
		{\Huge\bfseries Theoretische Methoden des Genitivs}\bigskip\\
		{\huge\bfseries \thesemester}\\
		\vspace{.5cm}
		\ifdisplaysolutions
			\vspace{-.25cm}
			{\Huge\bfseries \color{red}\textsc{Musterlösung}}\\
			\vspace{-.75cm}
		\fi

    {\Large
    	\vspace{1cm}
    	{\sffamily \theprofessor} \\
    	\vspace{0.5cm}
      %%% Settings
    	Technische Universität Dortmund \\
    	\vspace{1cm}
    	\begin{tabular}{l l}
      	Name:					& \underline{\hspace{8cm}} \\
      	Vorname:			& \underline{\hspace{8cm}} \\
      	Matrikel-Nr.: & \underline{\hspace{8cm}} \\
      	Studiengang:	& \underline{\hspace{8cm}} \\
      	Übungsgruppe: & \underline{\hspace{8cm}} \\
      	Unterschrift: & \underline{\hspace{8cm}}
      \end{tabular}
    }\\

    \vspace{0.5cm}
    \myuline{Erlaubte Hilfsmittel:} \\
    	Keine \\
    \myuline{Benutzung dokumentenechter Stifte (keine Bleistifte)! Keine Rotstifte!} \\
    \vspace{0.5cm}
    Die Bearbeitungszeit beträgt \textbf{180 Minuten}. \\
    \vspace{0.5cm}
    Trennen Sie unter keinen Umständen die Heftung der Klausur.
		Benutzen Sie nur die Blätter der Heftung (auch Rückseiten),
		oder durch die Assistenten ausgeteilte, zusätzliche Blätter. \\
    Schreiben Sie \myuline{Name und Vorname} oben auf jedes Lösungblatt. \\
    Legen Sie bitte zu Beginn der Klausur Ihren Studierendenausweis sichtbar aus. \\

    \begin{center}
    	\resizebox{0.7\textwidth}{!}{
      	\begin{tabular}{| c | | *{7}{p{1cm}|} | p{1cm} |}
        	\hline
          %%% Settings | die {7} ↑ auch | und die Anzahl an &
        	Aufgabe & 1 & 2 & 3 & 4 & 5 & 6 & $\sum$ \\
        	\hline
        	Max. Punkte & 18 & 10 & 10 & 6 & 6 & 6 & 56 \\
        	\hline
        	& & & & & & & \\
        	Erreichte Punkte & & & & & & & \\
        	& & & & & & & \\
					\hline
      	\end{tabular}
      } \\
      \vspace{.45cm}
      %%% Settings
      \SI{100}{\percent} entsprechen XX Punkten.
			Die Klausur gilt mit XX erreichten Punkten als bestanden! \\
      \begin{large} Viel Erfolg! \end{large} \\
      \vspace{.45cm}
			\thedate
    \end{center}
		\vspace{.75cm}
    \fbox{\textbf{Bitte unterschreiben Sie die Erklärungen auf der Rückseite.}}
  \end{center}
\end{titlepage}

\newpage
\thispagestyle{empty}
\begin{framed}
	\textbf{Die folgende Erklärung bitte \myuline{vor Klausurbeginn} lesen und unterschreiben:}
  \parskip1ex

	Ich habe zur Kenntnis genommen, dass die
  %%% Settings
  Fakultät Genitiv
  diese Klausur nur bis zum
  %%% Settings
  \textbf{19.05.2020}
  aufbewahrt,
  falls ich sie nicht nach der Klausureinsicht persönlich in Empfang nehme.
  Nach diesem Zeitpunkt werden nur das Deckblatt mit persönlichen Angaben und
	Klausurergebnis sowie diese Seite archiviert, die Aufgabenblätter werden vernichtet.
  \parskip1ex

  %%% Settings
  31. Februar 1234,~~Unterschrift: \underline{\hspace{12.1cm}}
\end{framed}

\begin{framed}
  \textbf{Die folgende Erklärung bitte \myuline{vor Klausurbeginn} lesen und unterschreiben:}
  \parskip1ex

  Ich bin damit einverstanden, dass meine Note unter Angabe meiner Matrikelnummer
	online im Moodle veröffentlicht wird.
  \parskip1ex

  %%% Settings
  31. Februar 1234,~~Unterschrift: \underline{\hspace{12.1cm}}
\end{framed}

\vfill

\begin{framed}
  \textbf{Die folgende Erklärung bitte \myuline{bei Klausurrückgabe} lesen und unterschreiben:}
  \parskip1ex

  Hiermit bestätige ich den Erhalt meiner vollständigen Klausur.
  Ich erkenne das Ergebnis an.
	\parskip1ex

  Datum:\underline{\hspace{3.5cm}},~~Unterschrift: \underline{\hspace{10.1cm}}
\end{framed}

\newpage

\begin{exercise}{Kurzfragen}{18}
  \begin{enumerate}
    \item Geben Sie ihren Lieblingsdinosaurier an!
      \hfill [2P]
      \ifdisplaysolutions
        \\ Lösung
      \else
        \vspace{4cm}
      \fi
  \end{enumerate}
\end{exercise}


\ifdisplaysolutions
\newpage
\else
\leerseite
\fi

\begin{exercise}{Aufgabe 2}{10}

\end{exercise}

\begin{solution}

\end{solution}

\ifdisplaysolutions
\newpage
\else
\leerseite
\fi

\end{document}
