\documentclass[
  paper=a4,
  headsepline,
  parskip=half,
  headheight=45pt,
]{scrartcl}

\usepackage[aux]{rerunfilecheck}

% deutsche Spracheinstellungen
\usepackage[main=ngerman]{babel}

\usepackage{amsmath}
\usepackage{amssymb}
\usepackage{mathtools}

\usepackage[
  version=4,
  math-greek=default,
  text-greek=default,
]{mhchem}

% Fonteinstellungen
\usepackage{fontspec}

\usepackage[
  math-style=ISO,    % \
  bold-style=ISO,    % |
  sans-style=italic, % | ISO-Standard folgen
  nabla=upright,     % |
  partial=upright,   % /
]{unicode-math}
\setmathfont{Latin Modern Math}

% richtige Anführungszeichen
\usepackage[autostyle]{csquotes}

% Zahlen und Einheiten
\usepackage[
  locale=DE,                   % deutsche Einstellungen
  separate-uncertainty=true,   % Immer Fehler mit \pm
  per-mode=symbol-or-fraction, % m/s im Text, sonst Brüche
]{siunitx}

% schöne Brüche im Text
\usepackage{xfrac}

\usepackage[
  labelfont=bf,        % Tabelle x: Abbildung y: ist jetzt fett
  font=small,          % Schrift etwas kleiner als Dokument
  width=0.9\textwidth, % maximale Breite einer Caption schmaler
]{caption}
\usepackage{subcaption}

\usepackage{graphicx}

% Standardplatzierung für Floats einstellen
\usepackage{float}
\usepackage{scrhack}
\floatplacement{figure}{htbp}
\floatplacement{table}{htbp}

% schöne Tabellen
\usepackage{booktabs}
\usepackage{multirow}

\usepackage{xparse}
\usepackage{xstring}
\usepackage{xcolor}


% Kopfzeile:
\usepackage{scrlayer-scrpage}


% Pakate für die Aufgaben
\usepackage{xsim}
\DeclareTranslation{german}{exsheets-exercise-name}{Aufgabe}
\usepackage{needspace}
\DeclareExerciseEnvironmentTemplate{runin}
{%
\par\vspace{\baselineskip}
\Needspace * {2\baselineskip}
\noindent
\textbf{\XSIMmixedcase{\GetExerciseName}~\GetExerciseProperty{counter}:}%
\GetExercisePropertyT{subtitle}{ \textit{#1}} %
\\ }
 {}
\xsimsetup{
exercise/within=section ,
exercise/template=runin,
solution/template=runin,
}

\usepackage{enumitem}
\setlist[enumerate,1]{label=\bfseries\alph*), parsep=0pt}

% Hyperlinks im Dokument
\usepackage[
  colorlinks,
  urlcolor=tudark,
  unicode,
  pdfcreator={},  % PDF-Attribute säubern
  pdfproducer={}, % "
]{hyperref}
\usepackage{bookmark}

% Trennung von Wörtern mit Strichen
\usepackage[shortcuts]{extdash}
\usepackage[paper=a4paper, bottom=20mm, right=27mm, left=27mm]{geometry}
\usepackage{tikz}

% german
\DeclareExerciseType{aufgabe}{
  exercise-env = aufgabe ,
  solution-env = loesung ,
  exercise-name = Aufgabe ,
  solution-name = Lösung ,
  exercises-name = Aufgaben ,
  solutions-name = Lösungen ,
  exercise-template = runin ,
  solution-template = runin ,
  exercise-heading = \subsection* ,
  solution-heading = \subsection*
}

%% activate the solutions if jobname is solution
\IfStrEq{\jobname}{\detokenize{solution}}{% True branch
  \xsimsetup{
    loesung/print=true,
  }
%  \geometry{paper=a4paper, bottom=17mm, top=20mm, left=17mm, right=10mm}
}{}

%%% change here for english or german or other languages
%%% or change color from tu settings
\setkomafont{pagehead}{\bfseries\upshape\large}
\makeatletter
\ihead{%
  \textcolor{tugreen}{\Large \sheetnumber. Übungsblatt} \\
  Abgabe: \handindate
}
\chead{}
\ohead{%
  \textcolor{tudark}{\@title~im \@date~} \\
  \textcolor{tudark}{\@author}
}
\makeatother

\renewcommand{\theequation}{\theaufgabe{}.\arabic{equation}}%
\numberwithin{equation}{aufgabe}%

\usepackage{mleftright}
\DeclarePairedDelimiter{\bra}{\langle}{\rvert}
\DeclarePairedDelimiter{\ket}{\lvert}{\rangle}
\DeclarePairedDelimiterX{\braket}[2]{\langle}{\rangle}{#1 \delimsize| #2}

\usepackage{expl3}
\ExplSyntaxOn
\NewDocumentCommand \dif {m} {\mathinner{\mathrm{d} #1}}
\NewDocumentCommand \del {m} {\mathinner{\mathrm{δ} #1}}
\NewDocumentCommand \Del {m} {\mathinner{\mathrm{Δ} #1}}
\NewDocumentCommand \matrize {m} {\underline{\underline{#1}}}
\NewDocumentCommand \abs {m} {\left|#1\right|}
\NewDocumentCommand \ableitung {mm} {\frac{\partial #1}{\partial #2}}
\NewDocumentCommand \abl {mm} {\frac{\dif{#1}}{\dif{#2}}}
\NewDocumentCommand \I {} {\mathrm{i}}
\NewDocumentCommand \E {} {\mathrm{e}}
\NewDocumentCommand \mpi {} {\symup{π}}
\NewDocumentCommand \grau {m} {\textcolor{gray}{#1}}
\NewDocumentCommand \rot {m} {\textcolor{red}{#1}}
\NewDocumentCommand \tug {m} {\textcolor{tugreen}{#1}}
\NewDocumentCommand \zB {} {z.\,B.~}
\NewDocumentCommand \DaH {} {d.\,h.~}
\NewDocumentCommand \IN {} {^{-1}}
\NewDocumentCommand \vn {} {\vec{\nabla}}
\let\ltext=\l
\RenewDocumentCommand \l {} {\TextOrMath{ \ltext }{ \mleft }}
\let\raccent=\r
\RenewDocumentCommand \r {} {\TextOrMath{ \raccent }{ \mright }}
\let\ktext=\k
\RenewDocumentCommand \k {m} {\TextOrMath{ \ltext }{ \l(#1\r) }}
\ExplSyntaxOff
\NewDocumentCommand \zZ {} {\textsf{Z\kern-.49em\raise-0.65ex\hbox{Z}}}

\xdefinecolor{tugreen}{RGB}{128, 186, 38}
\colorlet{tudark}{tugreen!60!black}

% See: https://alexwlchan.net/2017/10/latex-underlines/
\usepackage{contour}
\usepackage{ulem}
\renewcommand{\ULdepth}{1.8pt}
\contourlength{0.8pt}
\newcommand{\myuline}[1]{%
  \uline{\phantom{#1}}%
  \llap{\contour{white}{#1}}%
}

