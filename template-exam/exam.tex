\documentclass[fleqn,twoside]{scrartcl}
\usepackage[utf8]{inputenc}

\usepackage[aux]{rerunfilecheck}
\usepackage[main=ngerman]{babel}

\usepackage{amsmath}
\usepackage{amssymb}
\usepackage{mathtools}

\usepackage{fontspec}

\usepackage[
  math-style=ISO,    % \
  bold-style=ISO,    % |
  sans-style=italic, % | ISO-Standard folgen
  nabla=upright,     % |
  partial=upright,   % /
]{unicode-math}
\setmathfont{Latin Modern Math}

% richtige Anführungszeichen
\usepackage[autostyle]{csquotes}

% Zahlen und Einheiten
\usepackage[
  locale=DE,                   % deutsche Einstellungen
  separate-uncertainty=true,   % Immer Fehler mit \pm
  per-mode=symbol-or-fraction, % m/s im Text, sonst Brüche
]{siunitx}

% schöne Brüche im Text
\usepackage{xfrac}

\usepackage[
  labelfont=bf,        % Tabelle x: Abbildung y: ist jetzt fett
  font=small,          % Schrift etwas kleiner als Dokument
  width=0.9\textwidth, % maximale Breite einer Caption schmaler
]{caption}
\usepackage{subcaption}

\usepackage{graphicx}

% Standardplatzierung für Floats einstellen
\usepackage{float}
\usepackage{scrhack}
\floatplacement{figure}{htbp}
\floatplacement{table}{htbp}

% schöne Tabellen
\usepackage{booktabs}
\usepackage{multirow}

\usepackage{xparse}
\usepackage{xstring}
\usepackage{xcolor}
\xdefinecolor{tugreen}{RGB}{128, 186, 38}
\colorlet{tudark}{tugreen!60!black}

\usepackage{microtype}
% Margins
\usepackage[top=2cm,left=1.5cm,right=1.5cm]{geometry}

\usepackage{enumitem}
\setlist[enumerate,1]{label=\bfseries\alph*), parsep=0pt}

\usepackage[
  colorlinks,
  urlcolor=tudark,
  unicode,
  pdfcreator={},  % PDF-Attribute säubern
  pdfproducer={}, % "
]{hyperref}
\usepackage{bookmark}

\usepackage[shortcuts]{extdash}
\usepackage{tikz}
\usetikzlibrary{calc}
\usetikzlibrary{arrows}
\usetikzlibrary{arrows.meta}
\usetikzlibrary{patterns}
\usetikzlibrary{quotes,angles}
\usetikzlibrary{circuits.ee.IEC}
\usetikzlibrary{positioning}
\usetikzlibrary{decorations.pathmorphing}
\usetikzlibrary{decorations.pathreplacing}
\usepackage{framed}

% Head and Foot
\setkomafont{pageheadfoot}{\normalfont}
\markboth{}{Name: \underline{\hspace{8.1cm}} Matrikelnummer: \underline{\hspace{5.6cm}}}
\pagestyle{myheadings}

% Variables
%%% Settings
\newcommand{\thesemester}{Wintersemester 2019/2020}
\newcommand{\theprofessor}{Prof.~Dr.~Max~Muster}
\newcommand{\thedate}{??. Februar ????}

% Layout
\newcommand{\pkt}[1]{\textbf{#1\,P.}}

% Macros
\newcounter{exercise}
\newenvironment{exercise}
[2]
{\addtocounter{exercise}{1}{\Large{\bfseries{Aufgabe \arabic{exercise}:~#1}\hfill(#2 Punkte)}}\parskip5ex

}
{\medskip}
\newif\ifdisplaysolutions
% Switch on solutions
%\ifdefined\issolution
%\fi
\IfStrEq{\jobname}{\detokenize{solution}}{% True branch
  \displaysolutionstrue
}{}

\newenvironment{solution}
{
	\ifdisplaysolutions ~\\ {\Large{\bfseries{Lösung zu Aufgabe \arabic{exercise}:}\\\parskip5ex}}
  \else
  	\setbox0\vbox\bgroup
  \fi
}
{\ifdisplaysolutions
  \medskip
  \else
  	\egroup
  \fi
}

\setlength{\parindent}{0mm}

\usepackage{mleftright}
\DeclarePairedDelimiter{\bra}{\langle}{\rvert}
\DeclarePairedDelimiter{\ket}{\lvert}{\rangle}
\DeclarePairedDelimiterX{\braket}[2]{\langle}{\rangle}{#1 \delimsize| #2}

\usepackage{expl3}
\ExplSyntaxOn
\NewDocumentCommand \dif {m} {\mathinner{\mathrm{d} #1}}
\NewDocumentCommand \del {m} {\mathinner{\mathrm{δ} #1}}
\NewDocumentCommand \Del {m} {\mathinner{\mathrm{Δ} #1}}
\NewDocumentCommand \matrize {m} {\underline{\underline{#1}}}
\NewDocumentCommand \abs {m} {\left|#1\right|}
\NewDocumentCommand \ableitung {mm} {\frac{\partial #1}{\partial #2}}
\NewDocumentCommand \abl {mm} {\frac{\dif{#1}}{\dif{#2}}}
\NewDocumentCommand \I {} {\mathrm{i}}
\NewDocumentCommand \E {} {\mathrm{e}}
\NewDocumentCommand \mpi {} {\symup{π}}
\NewDocumentCommand \grau {m} {\textcolor{gray}{#1}}
\NewDocumentCommand \rot {m} {\textcolor{red}{#1}}
\NewDocumentCommand \tug {m} {\textcolor{tugreen}{#1}}
\NewDocumentCommand \zB {} {z.\,B.~}
\NewDocumentCommand \DaH {} {d.\,h.~}
\NewDocumentCommand \IN {} {^{-1}}
\NewDocumentCommand \punkte {m} {\textcolor{red}{\textit{(#1)}}}
\let\ltext=\l
\RenewDocumentCommand \l {} {\TextOrMath{ \ltext }{ \mleft }}
\let\raccent=\r
\RenewDocumentCommand \r {} {\TextOrMath{ \raccent }{ \mright }}
\ExplSyntaxOff
\NewDocumentCommand \zZ {} {\textsf{Z\kern-.49em\raise-0.65ex\hbox{Z}}}

% See: https://alexwlchan.net/2017/10/latex-underlines/
\usepackage{contour}
\usepackage{ulem}
\renewcommand{\ULdepth}{1.8pt}
\contourlength{0.8pt}
\newcommand{\myuline}[1]{%
  \uline{\phantom{#1}}%
  \llap{\contour{white}{#1}}%
}

\newcommand{\leerseite}{\newpage\hspace{1cm}\newpage}


\begin{document}
\DeclareGraphicsRule{*}{mps}{*}{}
\begin{titlepage}
	\begin{center}
		\vspace*{0cm}
		{\Huge\bfseries Abschlussklausur zur Vorlesung}\smallskip\\
    %%% Settings
		{\Huge\bfseries Modul}\bigskip\\
		{\huge\bfseries \thesemester}\\
		\vspace{.5cm}
		\ifdisplaysolutions
			\vspace{-.25cm}
			{\Huge\bfseries \color{red}\textsc{Musterlösung}}\\
			\vspace{-.75cm}
		\fi

    {\Large
    	\vspace{1cm}
    	{\sffamily \theprofessor} \\
    	\vspace{0.5cm}
      %%% Settings
    	Technische Universität Dortmund \\
    	\vspace{1cm}
    	\begin{tabular}{l l}
      	Name:					& \underline{\hspace{8cm}} \\
      	Vorname:			& \underline{\hspace{8cm}} \\
      	Matrikel-Nr.: & \underline{\hspace{8cm}} \\
      	Studiengang:	& \underline{\hspace{8cm}} \\
      	Übungsgruppe: & \underline{\hspace{8cm}} \\
      	Unterschrift: & \underline{\hspace{8cm}}
      \end{tabular}
    }\\

    \vspace{0.5cm}
    \myuline{Erlaubte Hilfsmittel:} \\
    %%% Settings
    	Keine \\
    \myuline{Benutzung dokumentenechter Stifte (keine Bleistifte)! Keine Rotstifte!} \\
    \vspace{0.5cm}
    %%% Settings
    Die Bearbeitungszeit beträgt \textbf{180 Minuten}. \\
    \vspace{0.5cm}
    Trennen Sie unter keinen Umständen die Heftung der Klausur.
		Benutzen Sie nur die Blätter der Heftung (auch Rückseiten),
		oder durch die Assistenten ausgeteilte, zusätzliche Blätter. \\
    Schreiben Sie \myuline{Name und Vorname} oben auf jedes Lösungblatt. \\
    Legen Sie bitte zu Beginn der Klausur Ihren Studierendenausweis sichtbar aus. \\

    \begin{center}
    	\resizebox{0.7\textwidth}{!}{
      %%% Settings
      %%% An Punkte anpassen
    	  \begin{tabular}{| c | | *{7}{p{1cm}|} | p{1cm} |}
        	\hline
        	Aufgabe & 1 & 2 & 3 & 4 & 5 & 6 & $\sum$ \\
        	\hline
        	Max. Punkte & 18 & 10 & 10 & 6 & 6 & 6 & 56 \\
        	\hline
        	& & & & & & & \\
        	Erreichte Punkte & & & & & & & \\
        	& & & & & & & \\
					\hline
      	\end{tabular}
      } \\
      \vspace{.45cm}
      %%% Settings
      \SI{100}{\percent} entsprechen 56 Punkten.
			Die Klausur gilt mit 29 erreichten Punkten als bestanden! \\
      \begin{large} Viel Erfolg! \end{large} \\
      \vspace{.45cm}
			\thedate
    \end{center}
		\vspace{.75cm}
    \fbox{\textbf{Bitte unterschreiben Sie die Erklärungen auf der Rückseite.}}
  \end{center}
\end{titlepage}

\newpage
\thispagestyle{empty}
\begin{framed}
	\textbf{Die folgende Erklärung bitte \myuline{vor Klausurbeginn} lesen und unterschreiben:}
  \parskip1ex

  %%% Settings
	Ich habe zur Kenntnis genommen, dass die Fakultät \punkte{\%\%\% Settings} diese Klausur nur
	bis zum \textbf{??.??.????} aufbewahrt, falls ich sie nicht nach der
	Klausureinsicht persönlich in Empfang nehme.
  Nach diesem Zeitpunkt werden nur das Deckblatt mit persönlichen Angaben und
	Klausurergebnis sowie diese Seite archiviert, die Aufgabenblätter werden vernichtet.
  \parskip1ex

  %%% Settings
  \thedate,~~Unterschrift: \underline{\hspace{12.1cm}}
\end{framed}

\begin{framed}
  \textbf{Die folgende Erklärung bitte \myuline{vor Klausurbeginn} lesen und unterschreiben:}
  \parskip1ex

  Ich bin damit einverstanden, dass meine Note unter Angabe meiner Matrikelnummer
	online im Moodle veröffentlicht wird.
  \parskip1ex

  %%% Settings
  \thedate,~~Unterschrift: \underline{\hspace{12.1cm}}
\end{framed}

\vfill

\begin{framed}
  \textbf{Die folgende Erklärung bitte \myuline{bei Klausurrückgabe} lesen und unterschreiben:}
  \parskip1ex

  Hiermit bestätige ich den Erhalt meiner vollständigen Klausur.
  Ich erkenne das Ergebnis an.
	\parskip1ex

  Datum:\underline{\hspace{3.5cm}},~~Unterschrift: \underline{\hspace{10.1cm}}
\end{framed}

\newpage

\begin{exercise}{Kurzfragen}{18}
  \begin{enumerate}
    \item Geben Sie ihren Lieblingsdinosaurier an!
      \hfill [2P]
      \ifdisplaysolutions
        \\ Lösung
      \else
        \vspace{4cm}
      \fi
  \end{enumerate}
\end{exercise}


\ifdisplaysolutions
\newpage
\else
\leerseite
\fi

\begin{exercise}{Name}{10}

\end{exercise}

\begin{solution}

\end{solution}

\ifdisplaysolutions
\newpage
\else
\leerseite
\fi

\end{document}
